% !TeX spellcheck = en_US
\chapter{Introduction}\label{introduction}

One important question for planet formation is, how water got to the earth. The part of the protoplanetary disk closest to the sun was too hot to make it possible that water can condense on Earth during formation. And while there are theories that the region where ice is possible inside the snow-line moved during Earth's formation\footcite{snowline}, the most popular theory is that water moved inwards in the solar system through collisions of water-rich proto-planets.%
\todo{citation needed}


\section{The perfect merging assumption}

To better understand how this process works, large n-body simulations over the lifetime of the solar systems have been conducted\footnote{for example \cite{dvorakSimulation}}. Most of these neglect the physical details of collisions when two bodies collide for simplicity and instead assume that a perfect merging occurs. So the entire mass of the two progenitor bodies and especially all of their water (ice) is retained in the newly created body. Obviously this is a simplification as in real collisions perfect merging is very rare and most of the time either partial accretion or a hit-and-run encounter occurs.\footcite{CollisionTypes} Therefore, the amount of water retained after collisions is consistently overestimated in these simulations. Depending on the parameters like impact angle and velocity, a large fraction of mass and water can be lost during collisions.\footcite{MaindlSummary}

\section{Some other heading}
\todo{find a name for this heading}

To understand how exactly the water transport works, one has to find an estimation of the mass and water fractions that are retained during two-body simulations depending on the parameters of the impact.
First, I will be shortly describing the simulation setup, the important parameters and the post-processing of the results (Chapter \ref{chapter:simulations}). Next I will summarize the results of the simulations and their properties (Chapter \ref{chapter:results}). In the main section I will then be describing three different approaches to interpolate and generalize these results for arbitrary collisions (Chapter \ref{chapter:interpolations}). 