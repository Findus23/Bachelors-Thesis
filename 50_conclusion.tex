\chapter{Comparison and Conclusion}
\label{sec:comparison}

All three methods for interpolation described above give results that follow the rough correlations from Section \ref{sec:cov}. So to compare them more precisely and measure their accuracy an additional set of 100 simulations (with the same properties as the ones listed in Section \ref{sec:resimulation}) was created. These results are neither used to train or select the neural network, nor are in the dataset for griddata and RBF interpolation. Therefore, we can use them to generate predictions for their parameters and compare them with the real fraction of water that remained in those simulations. By taking the mean absolute difference and the mean squared error between the predictions and the real result, the accuracy of the different methods can be estimated (Table \ref{tab:comparison}). As one of these parameter sets is outside the convex hull of the training data and griddata can't extrapolate, this simulation is skipped and only the remaining 99 simulations are considered for the griddata accuracy calculation. 

Of the three methods, the trained neural network has the highest mean squared error. This seems to be at least partly caused by the fact that during training of the neural network, the data is strongly generalized, causing the final network to output the \enquote{smoothest} interpolations. While this causes the errors to be higher, it might be possible that the fine structured details in the simulation output are just an artifact of the simulation setup and doesn't represent real world collisions.


Another important aspect to compare is the interpolation speed. The neural network is able to give the 100 results in about \SI{4}{\milli\second} (after loading the trained model which takes approximately one second). RBF interpolation is still reasonably fast, taking about \SI{8.5}{\second} (\SI{85}{\milli\second} per interpolation). But as \texttt{griddata} expects a grid-based parameter space, it becomes really slow when adding the resimulation data with random parameters. A single interpolation takes about \SI{35}{\second} totaling to about an hour for all 99 test cases. Using only the original dataset brings the run time down to around \SI{10}{\second}, but causes the results to be less accurate than all other methods. (first row in Table \ref{tab:comparison})

Interpolation using Radial Basis Functions all in all seems to be the most reliable method if there is enough input data and this input data is mostly spread randomly across the parameter space. It is easy to implement and quite fast to execute while still giving reasonable results. Neural Networks can also create realistic output, but have lots more configurable parameters that need to be tuned to get usable results. Their main advantage is, that in case the input set was by magnitudes larger, only the training would take longer, while evaluating the trained model wouldn't change.

To sum up, it is possible to estimate the amount of water lost in two-body-collisions with arbitrary collision parameters by simulating the outcome of a large amount of collisions using \texttt{SPH} and then doing linear interpolations to get results for parameters in between the ones from the simulation set. While the amount of remaining water is overestimated in this analysis as thermal effects during the collision are ignored, but the results are better than a perfect merging assumption.

