\input{template.tex}
\hypersetup{
	pdftitle={The Title of the Bachelor Thesis},
	pdfauthor={Lukas Winkler}
}
\title{The Title of the Bachelor Thesis}
\subtitle{bla bla}
\author{Lukas Winkler\footnote{\texttt{a01505981@unet.univie.ac.at}}}
%\publishers{TEST}
\date{1. August 2019}

\usepackage{lipsum}  % just for lorem ipsum
\newcommand{\blabla}{Bla bla bla}

\begin{document}
	
\maketitle

\tableofcontents

\chapter{Introduction}\label{introduction}


Here comes some short science explanation about how planets form and water is transported.
\lipsum[1]

\section{The perfect merging assumption}

To better understand how this process works, large n-body simulations over the lifetime of the solar systems have been conducted.\todo{give an example} Most of these neglect the physical details of collisions when two bodies collide for simplicity and instead assume that a perfect merging occurs. So all the mass of the two progenitor bodies and especially all of their water (ice) is retained in the newly created body. Obviously this is a simplification as in real collisions perfect merging is very rare and most of the time either partial accretion or a hit-and-run encounter occurs. (\cite{CollisionTypes}) Therefore the amount of water retained after collisions is consistently overestimated in these simulations. Depending on the parameters like impact angle and velocity a large fraction of mass and water can be lost during collisions.

\section{Some other heading}

To understand how the water transport works exactly one has to find an estimation of the mass and water fractions that are retained during two-body simulations depending on the parameters of the impact.

\todo{And here the explanation of the chapters}


\chapter{Simulations}

\section{Model}

For a realistic model of two gravitationally colliding bodies the SPH (smooth particle hydrodynamics) code \texttt{miluphCUDA} as explained in \cite{Schaefer2016} is used. It is able to simulate brittle failure and the interaction between multiple materials. 

In the simulation two celestrial bodies are placed far enough apart so that tidal forces can affect the collision. Both objects consist of a core with the physical properties of basalt rocks and a outer mantle made of water ice. 

To keep the simulation time short and make it possible to do many simulations with varying parameters 20k SPH particles are used\todo{Why 20k?} and each simulation is ran for 300 timesteps of each \SI{144}{\second} so that a whole day of collision is simulated.
%These will be split into two bodies according to the parameters of the simulations and placed far enough away that 

\section{Parameters}

Six parameters have been identified that have an influence on how the result of the simulation

\subsection{impact velocity}

The collision velocity $v_0$ is defined in units of the mutual escape velocity of the projectile and the target. Simulations have been made from $v_0=1$ to $v_0=5$. As one expects a higher velocity results in a stronger collision and more and smaller fragments.

\subsection{impact angle}

The impact angle is defined in a way that $\alpha=\ang{0}$ corresponds to a head-on collision and higher angles increase the chance of a hit-and-run encounter. The simulation ranges from $\alpha=\ang{0}$ to $\alpha=\ang{60}$

\subsection{target and projectile mass}
\todo{make sure I am not mixing up target and projectile here}

The masses in this simulation range from about two Ceres masses (\SI{1.88e+21}{\kilogram}) to about two earth masses (\SI{1.19e+25}{\kilogram}). In addition to the target mass $m$, the mass fraction between target and projectile $\gamma$ is defined. As the whole simulation setup is symmetrical between the two bodies only mass fractions below and equal to one have been considered.

\subsection{water fraction of target and projectile}

Both bodies 


\begin{table}
	\centering
	\begin{tabular}{r|rrrrr}
		$v_0$ & 1 & 1.5 & 2&3 & 5 \\
		$\alpha$ & \ang{0} & \ang{20} & \ang{40} & \ang{60} &\\
		$m$ &\SI{e21}{\kilogram} & \SI{e23}{\kilogram} & \SI{e24}{\kilogram} & \SI{e25}{\kilogram} &\\
		$\gamma$ & 0.1 & 0.5 & 1 &&\\
		water fraction target & \SI{10}{\percent} & \SI{20}{\percent} &&&\\		
		water fraction projectile & \SI{10}{\percent} & \SI{20}{\percent} &&&\\
	\end{tabular}
	\label{tab:first_simulation_parameters}
	\caption{parameter set of the first simulation run}
\end{table}

\section{More text}

\lipsum[2-5]



\begin{itemize}
\setlength\itemsep{-0.5em}
\item test
\item more test
\end{itemize}

See Chapter  \ref{introduction}.


\section{An equation}

\begin{align}
	a&=\sqrt{4} \\
	b&=a^2
\end{align}

% ----------------------------------------------------------

\appendix
\chapter{Some data}

\lipsum[1]\footcite{Schaefer2016}\footcite{dvorakMoon}\footcite{MaindlSummary}\footcite{Burger2018}\footcite{Dorninger}\footcite{CollisionParameters}\footcite{CollisionTypes}

\printbibliography


\end{document}

