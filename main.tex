\input{template.tex}
\hypersetup{
	pdftitle={The Title of the Bachelor Thesis},
	pdfauthor={Lukas Winkler}
}
\title{The Title of the Bachelor Thesis}
\subtitle{bla bla}
\author{Lukas Winkler\footnote{\texttt{a01505981@unet.univie.ac.at}}}
%\publishers{TEST}
\date{1. August 2019}

\usepackage{lipsum}  % just for lorem ipsum
\newcommand{\blabla}{Bla bla bla}

\begin{document}
	
\maketitle

\tableofcontents

\chapter{Introduction}\label{introduction}


Here comes some short science explanation about how planets form and water is transported.
\lipsum[1]

\section{The perfect merging assumption}

To better understand how this process works, large n-body simulations over the lifetime of the solar systems have been conducted.\todo{give an example} Most of these neglect the physical details of collisions when two bodies collide for simplicity and instead assume that a perfect merging occurs. So all of the mass of the two progenitor bodies and especially all of their water is retained in the newly created body. Obviously this is a simplification as in real collisions perfect merging is very rare and most of the time either partial accretion or a hit-and-run encounter occurs. (\cite{CollisionTypes}) Therefore the amount of water retained after collisions is consistently overestimated in these simulations.



\begin{figure}
	\centering
%	\includegraphics[width=.5\linewidth]{thefile.png}
	\caption{\blabla}
	\label{fig:bla}
\end{figure}

\chapter{Simulations}

\lipsum[1-2]

Even more \enquote{Text} with \SI{20.4e5}{\kilo\meter\per\hour}!

\begin{align}
	\dv{a}
\end{align}

\section{More text}

\lipsum[2-5]



\begin{itemize}
\setlength\itemsep{-0.5em}
\item test
\item more test
\end{itemize}

See Chapter  \ref{introduction}.


\section{An equation}

\begin{align}
	a&=\sqrt{4} \\
	b&=a^2
\end{align}

% ----------------------------------------------------------

\appendix
\chapter{Some data}

\lipsum[1]\footcite{Schaefer2016}\footcite{dvorakMoon}\footcite{MaindlSummary}\footcite{Burger2018}\footcite{Dorninger}\footcite{CollisionParameters}\footcite{CollisionTypes}

\printbibliography


\end{document}

